\documentclass{article}
\usepackage[a4paper]{geometry}
\usepackage{listings}
\usepackage{color}
\geometry{left=2.54cm,right=2.54cm,top=1.0cm,bottom=1.0cm} 

\usepackage{amssymb}
\usepackage{pythonhighlight}
\UseRawInputEncoding

\usepackage{xcolor}

\lstset{
    language=Python,
    commentstyle=\color{red}, % 设置注释的样式为红色
    breaklines=true,         % 自动换行
    breakatwhitespace=false, % 只在空格处换行
    frame=single,            % 将代码框起来
}




\title{Numerische Implementierung der linearen FEM\\Aufgabenblatt 1}
\author{Group: diamond\\
        Yahuan Shi/494877\\
        Chih-Cheng Huang/ 0494886\\
        Xingyu Shang/ 498775\\
        Haowang Zhang/ 498759\\ 
       }
\date{April 2024}

\begin{document}

\maketitle



\section{Aufgabe 1}
\subsection{The six lower-pair joints (ignore joint limits).}

\begin{lstlisting}[language=Matlab]
x = 1 + 2 + 3 + 4*2
y = 1:5
z = x + y;
z2 = x + y

x = 1:10
y = x.^2
plot (x, y)
plot (x, y, 'go ')
plot (x, y, 'rx -')
\end{lstlisting}
Q: Was ergibt der Doppelpunkt-Operator?\\
\textcolor{red}{A}: $y=1:5$  erstellt eine Sequenz von 1 bis 5 in Schritten von 1. Daher ist der Wert von y [1, 2, 3, 4, 5].\\
x = 1:10 erstellt eine Sequenz von 1 bis 10 in Schritten von 1. Daher ist der Wert von x [1, 2, 3, 4, 5, 6, 7, 8, 9, 10].\\

\begin{lstlisting}[language=Matlab]
s = [1 2 3]
t = [1; 2; 3]

A = [1 2 3; 4 5 6; 7 8 9]
A*t

A*s
s'
A'
A*s'

t*s
s*t

s' .* t

A^2
A.^2
\end{lstlisting}
Q: Welche Funktion hat das Semikolon beim Anlegen einer Matrix/ eines Vektors?\\
\textcolor{red}{A}: Das Semikolon wird verwendet, um Zeilen in der Matrix zu trennen. 
Q: Welche Funktion hat es am Ende einer Zeile?\\
\textcolor{red}{A}: Ein Semikolon beendet eine Zeile und weist MATLAB an, die Ergebnisse dieser Zeile nicht im Befehlsfenster auszugeben. Dies wird oft als „Ausgabe unterdrücken“ bezeichnet.
Q:Welche Funktion hat der Apostroph als Operator?\\
\textcolor{red}{A}: Der Apostroph wird verwendet, um die Transponierte einer Matrix oder eines Vektors zu erstellen. 
Q:Welche Funktion hat der Punkt als Operator?\\
\textcolor{red}{A}: Der Punkt wird verwendet, um elementweise Operationen auf Arrays durchzuführen. \\\\
Statt der Definition von Matrizen und Vektoren von Hand knnen die folgenden Funktionen zum Erzeugen verwendet werden. Vektoren k¨onnen dabei wie Matrizen mit nur einer Spalte erzeugt werden. Beschreiben Sie jeweils die konkrete Funktion in Ihren eigenen Worten:\\

\begin{lstlisting}[language=Matlab]
n_row = 10
n_col = 5
M = zeros (n_row , n_col )
%Diese Funktion erzeugt eine Matrix mit der angegebenen Anzahl von Zeilen (n_row)
und Spalten (n_col), wobei alle Elemente auf Null gesetzt werden.
I = eye( n_row )
%Diese Funktion erzeugt eine Einheitsmatrix mit der angegebenen Anzahl von 
Zeilen (n_row) und Spalten (n_row).
%Die Diagonalelemente der Einheitsmatrix sind Eins, während alle anderen Elemente Null sind.
U = ones (n_row , n_col )
%Diese Funktion erzeugt eine Matrix mit der angegebenen Anzahl von Zeilen (n_row) und Spalten (n_col), wobei alle Elemente auf Eins gesetzt werden.
R = rand (n_row , n_col )
%Diese Funktion erzeugt eine Matrix mit der angegebenen Anzahl von Zeilen (n_row) und Spalten (n_col), wobei die Elemente zufällige Werte zwischen 0 und 1 haben.
D = diag (1: n_col )
%Diese Funktion erzeugt eine diagonale Matrix D mit den Zahlen 1 bis n_col auf der Hauptdiagonale.
D2 = D + diag (1:( n_col -2) , 2)
%Diese Funktion erzeugt eine diagonale Matrix D2, bei der die Zahlen 1 bis n_col-2 auf der zweiten Nebendiagonale platziert werden.

E = U + R
\end{lstlisting}


\section{Aufgabe 2}
Solution:\\





\end{document}

